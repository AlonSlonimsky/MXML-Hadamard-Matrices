\documentclass{beamer}
\usetheme{Copenhagen}
\usepackage{graphicx} % Required for inserting images
\setbeamertemplate{navigation symbols}{}

\usepackage{amssymb}

\title{Complex Hadamard Matrices}
\author{Alon Slonimsky}%put your name here too
\date{Fall 2024}

\newcommand{\R}{\mathbb{R}}
\newcommand{\C}{\mathbb{C}}
\newcommand{\U}{\mathbb{U}}
\newcommand{\ip}[2]{\langle #1, #2 \rangle}
\newcommand{\RHM}{Real Hadamard Matrix}
\newcommand{\RHMs}{Real Hadamard Matrices}
\newcommand{\CHM}{Complex Hadamard Matrix}
\newcommand{\CHMs}{Complex Hadamard Matrices}

\newtheorem{remark}{Remark}

\begin{document}

\maketitle
\begin{frame}{Definitions:}
    \begin{definition}[Hermitian Adjoint]
        The Hermitian Adjoint $M^*$ of a Matrix $M\in \C^{n\times n}$ is the transpose of $M$, followed by the inverse of every element.
    \end{definition}
    \begin{definition}[Hermitian Inner Product for $\C^n$]
        For $\vec x,\vec y\in\C^n, \ip{\vec x}{\vec y}:=\sum_{i=1}^nx_iy_i^{-1}$
    \end{definition}
\end{frame}
\begin{frame}{Real Hadamard Matrices}
    \begin{definition}[Real Hadamard Matrix]
        A Matrix $M\in\{\pm1\}^{n\times n}$ is a Real Hadamard Matrix if $MM^T=nI_{n\times n}$
    \end{definition}
    \pause
    \begin{definition}[Real Hadamard Matrix]
        Equivalently, a Matrix $M\in\{\pm1\}^{n\times n}$ is a Real Hadamard Matrix if every pair of distinct rows is orthogonal under the dot product.
    \end{definition}
\end{frame}
\begin{frame}{Complex Hadamard Matrices}
    \begin{definition}[Set of Complex Units]
        Let $\U:=\{z=a+bi:a,b\in\R\land|z|=1\}$, the set of complex units.
    \end{definition}
    \pause
    \begin{definition}[Complex Hadamard Matrix]
        A Matrix $M\in\U^{n\times n}$ is a Complex Hadamard Matrix if $MM^*=nI_{n\times n}$
    \end{definition}
    \pause
    \begin{definition}[Complex Hadamard Matrix]
        Equivalently, a Matrix $M\in\U^{n\times n}$ is a Complex Hadamard Matrix if every pair of distinct rows $r_i,r_j$ has that $\ip{r_i}{r_j}=0$.
    \end{definition}
    \pause
    \begin{remark}
        The definition of \CHMs \ generalizes the definition of \RHMs
    \end{remark}
\end{frame}
\begin{frame}{Definitions}
    \begin{definition}[Trade]
        
    \end{definition}
    \pause
    \begin{definition}[Switch]
        
    \end{definition}
    \pause
    \begin{remark}[Log Form of a Switch]
        
    \end{remark}
\end{frame}
\begin{frame}{Definitions}
    \begin{definition}[N-Uniform Switch]
        
    \end{definition}
    \pause
    \begin{remark}[N-Uniform Row/Column]
        
    \end{remark}
\end{frame}
\begin{frame}{Lemma 3 (\'{O} Cath\'{a}in and Wanless' trades paper [OW])}%must also add a citation here
    \begin{lemma}[Lemma 3]
        
    \end{lemma}
\end{frame}

\begin{frame}{Ratio of Inner Product of N-Uniform Row}
    \begin{lemma}%I need to rewrite this, it is difficult to read and also assumes some existing understanding of the proof of lemma 3
        For $N\geq2$ define a N-uniform trade $T$ in UH. Let $K=\{k_1,k_2,\dots k_N\}$ $(1\notin K)$ be the list of scalars used in the N-uniform trade. Let $T_i$ be the set of elements in $T$ that are multiplied by $k_i$ in the trade. Let $R:=\overline{T}$ be the remaining elements. Then we have that $T=\bigsqcup_{i=1}^NT_i$ (disjoint union). Let $r_a$ be a row such that an element of $r_a$ is in each of $T_i$. Let $r_b$ be a row that does not participate in the trade. Let $r_{a,A}$ be the elements of $r_a$ that are in set $A$.\\
    Then $\langle r_{a,T_1},r_{b,T_1}\rangle=\sum_{i=2}^N\frac{k_i-1}{1-k_1}\langle r_{a,T_i},r_{b,T_i}\rangle$
    \end{lemma}
\end{frame}

\begin{frame}{Ratio of Inner Product of N-Uniform Row}
    \begin{proof}
        By orthogonality in both the original matrix and the traded matrix, we have that: 
    $$\sum_{i=1}^N\langle r_{a,T_i},r_{b,T_i}\rangle + \langle r_{a,R},r_{b,R}\rangle=0$$
    And also:
    $$\sum_{i=1}^N\langle k_ir_{a,T_i},r_{b,T_i}\rangle + \langle r_{a,R},r_{b,R}\rangle=\sum_{i=1}^Nk_i\langle r_{a,T_i},r_{b,T_i}\rangle + \langle r_{a,R},r_{b,R}\rangle=0$$
    Then as both values are equal, we remove the common term and see:
    $$\sum_{i=1}^N\langle r_{a,T_i},r_{b,T_i}\rangle = \sum_{i=1}^Nk_i\langle r_{a,T_i},r_{b,T_i}\rangle$$
    \end{proof}
\end{frame}
\begin{frame}{Ratio of Inner Product of N-Uniform Row}
\begin{proof}
    $$\sum_{i=1}^N\langle r_{a,T_i},r_{b,T_i}\rangle = \sum_{i=1}^Nk_i\langle r_{a,T_i},r_{b,T_i}\rangle$$
    So by collecting like terms, we see:
    \begin{align*}
        (1-k_1)\langle r_{a,T_i},r_{b,T_i}\rangle&=\sum_{i=2}^N(k_i-1)\langle r_{a,T_i},r_{b,T_i}\rangle\\\implies \langle r_{a,T_1},r_{b,T_1}\rangle&=\sum_{i=2}^N\frac{k_i-1}{1-k_1}\langle r_{a,T_i},r_{b,T_i}\rangle
    \end{align*}
\end{proof}
\end{frame}


\begin{frame}{Theorem 4 (OW)}%must also add a citation here
    \begin{theorem}[Theorem 4]
        
    \end{theorem}
\end{frame}
\end{document}
