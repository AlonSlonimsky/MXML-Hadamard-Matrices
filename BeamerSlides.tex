\documentclass{beamer}
\usetheme{Copenhagen}
\usepackage{graphicx} % Required for inserting images
\setbeamertemplate{navigation symbols}{}

\usepackage{amssymb}

\title{Complex Hadamard Matrices}
\author{Alon Slonimsky}%put your name here too
\date{Fall 2024}

\newcommand{\R}{\mathbb{R}}
\newcommand{\C}{\mathbb{C}}
\newcommand{\U}{\mathbb{U}}
\newcommand{\ip}[2]{\langle #1, #2 \rangle}
\newcommand{\RHM}{Real Hadamard Matrix}
\newcommand{\RHMs}{Real Hadamard Matrices}
\newcommand{\CHM}{Complex Hadamard Matrix}
\newcommand{\CHMs}{Complex Hadamard Matrices}

\newtheorem{remark}{Remark}

\begin{document}

\maketitle
\begin{frame}{Definitions:}
    \begin{definition}[Set of Complex Units]
        Let $\U:=\{z=a+bi:a,b\in\R\land|z|=1\}$, the set of complex units.
    \end{definition}
    \pause
    \begin{definition}[Hermitian Adjoint]
        The Hermitian Adjoint $M^*$ of a Matrix $M\in \C^{n\times n}$ is the transpose of $M$, followed by the conjugate of every element.
    \end{definition}
    \begin{remark}
        $\forall x\in\U,\overline{x}=x^{-1}$.\\ The inverse and the conjugate of a complex unit are always equal.
    \end{remark}
    \pause
    \begin{definition}[Standard Hermitian Inner Product for $\C^n$]
        For $\vec x,\vec y\in\C^n, \ip{\vec x}{\vec y}:=\sum_{i=1}^nx_i\overline{y_i}$
    \end{definition}
\end{frame}
\begin{frame}{Real Hadamard Matrices}
    \begin{definition}[Real Hadamard Matrix]
        A Matrix $M\in\{\pm1\}^{n\times n}$ is a Real Hadamard Matrix if $MM^T=nI_{n\times n}$
    \end{definition}
    \pause
    \begin{definition}[Real Hadamard Matrix]
        Equivalently, a Matrix $M\in\{\pm1\}^{n\times n}$ is a Real Hadamard Matrix if every pair of distinct rows is orthogonal under the dot product.
    \end{definition}
\end{frame}
\begin{frame}{Complex Hadamard Matrices}
    \begin{definition}[Complex Hadamard Matrix]
        A Matrix $M\in\U^{n\times n}$ is a Complex Hadamard Matrix if $MM^*=nI_{n\times n}$
    \end{definition}
    \pause
    \begin{definition}[Complex Hadamard Matrix]
        Equivalently, a Matrix $M\in\U^{n\times n}$ is a Complex Hadamard Matrix if every pair of distinct rows $r_i,r_j$ has that $\ip{r_i}{r_j}=0$.
    \end{definition}
    \pause
    \begin{remark}
        The definition of \CHMs \ generalizes the definition of \RHMs
    \end{remark}
\end{frame}
\begin{frame}{Definitions}
    \begin{definition}[Trade]
        
    \end{definition}
    \pause
    \begin{definition}[Switch]
        
    \end{definition}
    \pause
    \begin{remark}[Log Form of a Switch]
        
    \end{remark}
\end{frame}
\begin{frame}{Lemma 3 (\'{O} Cath\'{a}in and Wanless' trades paper [OW])}
    
\end{frame}
\begin{frame}{Theorem 4 (OW)}
    
\end{frame}
\end{document}
